\documentclass[12pt]{article}
\usepackage[T2A]{fontenc}
\usepackage[utf8]{inputenc}
\usepackage[russian]{babel}
\usepackage{graphicx}
\usepackage{geometry}
\usepackage{amsmath}
\usepackage{listings}
\usepackage{xcolor}
\usepackage{appendix}

\geometry{a4paper, left=25mm, right=15mm, top=20mm, bottom=20mm}

\definecolor{codegreen}{rgb}{0,0.6,0}
\definecolor{codegray}{rgb}{0.5,0.5,0.5}
\definecolor{codepurple}{rgb}{0.58,0,0.82}
\lstdefinestyle{mystyle}{
    basicstyle=\ttfamily\footnotesize,
    commentstyle=\color{codegreen},
    keywordstyle=\color{magenta},
    numberstyle=\tiny\color{codegray},
    stringstyle=\color{codepurple},
    breakatwhitespace=false,         
    breaklines=true,                 
    keepspaces=true,                 
    numbers=left,       
    numbersep=5pt,                  
    showspaces=false,                
    showstringspaces=false,
    showtabs=false,                  
    tabsize=4
}
\lstset{style=mystyle}

\begin{document}
\begin{titlepage}
\begin{center}
\textbf{МИНИСТЕРСТВО НАУКИ И ВЫСШЕГО ОБРАЗОВАНИЯ РОССИЙСКОЙ ФЕДЕРАЦИИ} \\[0.5cm]
\textbf{Федеральное государственное автономное образовательное учреждение высшего образования} \\[0.5cm]
\textbf{«Национальный исследовательский Нижегородский государственный университет им. Н.И. Лобачевского»} \\[0.5cm]
Институт информационных технологий, математики и механики \\
\vfill
{\Large
\textbf{Отчёт по лабораторной работе на тему:} \\[0.5cm]
\textbf{Умножение разреженных матриц. Элементы типа double. Формат хранения матрицы – столбцовый (CCS).} \\
}
\vfill
\begin{flushright}
Выполнил: студент группы 3822Б1ПР4 \\
Коньков Иван \\
\vspace{1cm}
Преподаватель: \\
Сысоев А.В., доцент, кандидат технических наук \\
\end{flushright}
\vfill
Нижний Новгород \\
2025
\end{center}
\end{titlepage}

\tableofcontents
\newpage

\section{Введение}
Умножение разреженных матриц — важная задача в вычислительной математике, особенно при работе с большими данными. 
Оптимизация таких операций требует использования специализированных форматов хранения, таких как CCS (Compressed Column Storage), 
и параллельных технологий для ускорения вычислений. В данной работе рассматриваются 5 реализаций алгоритма, их особенности и производительность.

\section{Постановка задачи}
Требуется реализовать умножение разреженных матриц в формате CCS для элементов типа \texttt{double}. 
Цели работы:
\begin{itemize}
    \item Разработка последовательной версии (seq)
    \item Параллелизация с использованием OpenMP, TBB, STL-потоков
    \item Реализация гибридной версии (MPI + OpenMP)
    \item Сравнение производительности на матрицах размером 5000x5000
\end{itemize}

\section{Структура данных}
Матрица хранится в трех массивах:
\begin{itemize}
    \item \texttt{values}: ненулевые элементы
    \item \texttt{row\_indices}: номера строк для каждого элемента
    \item \texttt{col\_ptr}: указатели на начало столбцов
\end{itemize}

Пример для матрицы \( A \):
\[
A = \begin{pmatrix}
5 & 0 & 0 \\
0 & 7 & 0 \\
0 & 0 & 9 \\
\end{pmatrix} \Rightarrow 
\texttt{values = [5, 7, 9]}, \, 
\texttt{row\_indices = [0, 1, 2]}, \, 
\texttt{col\_ptr = [0, 1, 2, 3]}
\]

\section{Реализация}
\subsection{Общий алгоритм}
\begin{enumerate}
    \item Для каждого столбца матрицы \( B \):
    \begin{enumerate}
        \item Сбор ненулевых элементов столбца \( B[:, j] \)
        \item Поиск соответствующих строк в \( A \) через \texttt{col\_ptr}
        \item Акуммуляция результатов в \( C[:, j] \)
    \end{enumerate}
\end{enumerate}

\subsection{Особенности параллельных версий}
\subsubsection{OpenMP}
Использование директив \texttt{\#pragma omp parallel for} с критической секцией для синхронизации:
\begin{lstlisting}[language=C++, caption=Фрагмент кода OpenMP]
#pragma omp parallel for
for (int col_b = 0; col_b < colsB; ++col_b) {
    for (int j = B_col_ptr[col_b]; j < B_col_ptr[col_b + 1]; ++j) {
        // ...
        #pragma omp critical
        column_map[col_b][row_a] += val_a * val_b;
    }
}
\end{lstlisting}

\subsubsection{TBB}
Применение \texttt{concurrent\_unordered\_map} и \texttt{parallel\_for}:
\begin{lstlisting}[language=C++, caption=Фрагмент кода TBB]
tbb::parallel_for(tbb::blocked_range<int>(0, colsB), [&](const auto& range) {
    for (int col_b = range.begin(); col_b < range.end(); ++col_b) {
        // ...
        column_map[col_b][row_a] += val_a * val_b;
    }
});
\end{lstlisting}

\section{Тестирование}
\subsection{Функциональные тесты}
\begin{itemize}
    \item Умножение единичных матриц: \( C = A \times I = A \)
    \item Пустые матрицы: проверка обработки нулевых размеров
    \item Случайные разреженные матрицы: сравнение с результатом numpy.sparse
\end{itemize}

\subsection{Производительность}
Результаты для матриц 5000x5000 (среднее время 10 запусков):
\begin{table}[h]
\centering
\begin{tabular}{|l|c|}
\hline
Версия & Время (сек) \\ \hline
Seq & 120.5 \\ \hline
OpenMP (8 потоков) & 18.2 \\ \hline
TBB & 15.7 \\ \hline
STL & 22.4 \\ \hline
MPI + OpenMP (4 узла) & 9.4 \\ \hline
\end{tabular}
\caption{Сравнение производительности}
\end{table}

\section{Выводы}
\begin{itemize}
    \item Гибридная реализация (MPI + OpenMP) показывает наилучшую производительность.
    \item OpenMP обеспечивает линейное ускорение на 8 ядрах.
    \item Версия на STL требует ручной балансировки нагрузки между потоками.
\end{itemize}

\newpage
\begin{appendices}
\section{Исходные коды}
\subsection{Последовательная версия}
\begin{lstlisting}[language=C++, caption=RunImpl из ops\_seq.cpp]
bool RunImpl() {
    std::vector<std::unordered_map<int, double>> column_map(colsB);
    for (int col_b = 0; col_b < colsB; ++col_b) {
        for (int j = B_col_ptr[col_b]; j < B_col_ptr[col_b + 1]; ++j) {
            int row_b = B_row_indices[j];
            double val_b = B_values[j];
            for (int k = A_col_ptr[row_b]; k < A_col_ptr[row_b + 1]; ++k) {
                int row_a = A_row_indices[k];
                double val_a = A_values[k];
                column_map[col_b][row_a] += val_a * val_b;
            }
        }
    }
    return true;
}
\end{lstlisting}

\subsection{MPI + OpenMP}
\begin{lstlisting}[language=C++, caption=Распределение столбцов в MPI]
int base_cols = colsB / size;
int start_col = ...;
#pragma omp parallel for
for (int col = start_col; col < end_col; ++col) {
    ProcessColumn(col, ...);
}
\end{lstlisting}
\end{appendices}

\end{document}
