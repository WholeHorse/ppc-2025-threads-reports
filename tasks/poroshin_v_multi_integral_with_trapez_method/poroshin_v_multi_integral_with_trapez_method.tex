\documentclass[12pt]{article}

\usepackage[T2A]{fontenc}
\usepackage[utf8]{inputenc}
\usepackage[russian]{babel}

\usepackage{amsmath,amssymb}
\usepackage{geometry}
\usepackage{graphicx}
\usepackage{listings}
\usepackage{hyperref}
\geometry{left=2cm,right=2cm,top=2cm,bottom=2cm}

\lstset{
  basicstyle=\small\ttfamily,
  breaklines=true,
  tabsize=2,
  language=C++,
  numbers=left,
  numberstyle=\tiny
}

\begin{document}

\begin{titlepage}
    \begin{center}
        \large 
        \textbf{ННГУ им. Лобачевского / ИИТММ / ПМИ}\\[0.5cm]

        \vspace{4cm}
        \textbf{\Large Отчёт по выполнению задания}\\
        \textbf{\large «Параллельное вычисление многомерных интегралов с использованием многошаговой схемы (метод трапеций)»}\\[3cm]

        \vspace{3cm}
        \textbf{Выполнил:}\\
        студент группы 3822Б1ПМоп3 \\
        \textit{Порошин Владислав Сергеевич}\\[1cm]

        \textbf{Преподаватель:}\\
        \textit{Сысоев Александр Владимирович, доцент}\\[2cm]

        \vfill
        \textbf{Нижний Новгород, 2025 г.}
    \end{center}
\end{titlepage}

\tableofcontents
\newpage

\section{Введение}
Вычисление многомерных интегралов является фундаментальной задачей вычислительной математики, имеющей широчайшее применение в физике (квантовая механика, статистическая физика), технике (анализ напряжений, электродинамика), экономике (оценка финансовых рисков) и многих других областях.

\section{Цель и задачи}
\subsection*{Цель работы}
Реализовать и сравнить несколько версий параллельного алгоритма вычислений многомерных интегралов с использованием многошаговой схемы (метод трапеций): SEQ, OpenMP, TBB, STL Threads, MPI+OMP.

\subsection*{Задачи}
\begin{itemize}
  \item Реализовать последовательный вариант алгоритма
  \item Реализовать четыре параллельных варианта
  \item Сравнить производительность и масштабируемость
  \item Сделать выводы о наиболее эффективном подходе
\end{itemize}

\section{Математическая формулировка алгоритма вычислений многомерных интегралов с использованием многошаговой схемы (метод трапеций)}
Пусть требуется вычислить $d$-мерный интеграл:
\[
I = \int_{a_1}^{b_1} \int_{a_2}^{b_2} \cdots \int_{a_d}^{b_d} f(\mathbf{x})  dx_1 dx_2 \cdots dx_d
\]
где $\mathbf{x} = (x_1, x_2, \dots, x_d)$.

\subsection*{Шаг 1: Разбиение области}
Для каждого измерения $k = 1, 2, \dots, d$:
\begin{itemize}
    \item Задаём число разбиений $n_k \geq 1$
    \item Вычисляем шаг интегрирования:
    \[
    h_k = \frac{b_k - a_k}{n_k}
    \]
    \item Формируем сетку узлов:
    \[
    x_k^{(i_k)} = a_k + i_k \cdot h_k, \quad i_k = 0, 1, \dots, n_k
    \]
\end{itemize}

\subsection*{Шаг 2: Определение весовых коэффициентов}
Для каждой координатной оси $k$ вводим весовую функцию:
\[
w_k(i_k) = 
\begin{cases} 
\frac{1}{2} & \text{если } i_k = 0 \text{ или } i_k = n_k \\
1 & \text{иначе}
\end{cases}
\]

\subsection*{Шаг 3: Многомерная аппроксимация}
Интеграл аппроксимируется составной формулой трапеций:
\[
I \approx T_d(f) = \prod_{k=1}^{d} h_k \cdot \sum_{i_1=0}^{n_1} \sum_{i_2=0}^{n_2} \cdots \sum_{i_d=0}^{n_d} 
\left[ \prod_{k=1}^{d} w_k(i_k) \right] f\left(x_1^{(i_1)}, x_2^{(i_2)}, \dots, x_d^{(i_d)}\right)
\]

\subsection*{Упрощённая запись}
Обозначим:
\begin{align*}
\mathbf{i} &= (i_1, i_2, \dots, i_d), \\
w(\mathbf{i}) &= \prod_{k=1}^{d} w_k(i_k), \\
\mathbf{x}^{(i)} &= \left( x_1^{(i_1)}, x_2^{(i_2)}, \dots, x_d^{(i_d)} \right)
\end{align*}
Тогда формула принимает компактный вид:
\[
\boxed{
T_d(f) = \left( \prod_{k=1}^{d} h_k \right) \cdot \sum_{\mathbf{i}} w(\mathbf{i})  f(\mathbf{x}^{(\mathbf{i})})
}
\]
где сумма берётся по всем мультииндексам $\mathbf{i} \in [0, n_1] \times [0, n_2] \times \cdots \times [0, n_d]$.

\section{Общие функции и структура реализации}
Во всех версиях алгоритма реализован единый интерфейс, включающий в себя следующие методы и вспомогательные функции.

\subsection*{PreProcessingImpl()}
Подготавливает входные данные. Извлекает параметры интегрирования - количество разбиений и границы по каждой размерности. Инициализирует аккумулятор результата, сами данные приводит к формату, удобному для обработки.

\subsection*{ValidationImpl()}
Проверяет корректность входных данных. Основные критерии: размер массива разбиений совпадает с размерностью пространства, массив разбиений не пуст, выходной буфер рассчитан только на 1 элемент (скалярный результат).

\subsection*{RunImpl()}
Основная логика запуска алгоритма. Включает вызов функции \texttt{CountMultiIntegralTrapezMethodStl}.

\subsection*{PostProcessingImpl()}
Формирует выходное значение в формате, ожидаемом системой. Переписывает результат из внутренней переменной в буфер \texttt{task\_data->outputs[0][0]}.

\subsection*{CalculateData()}
Реализует подготовительный этап для составной формулы трапеций, определяет: расстояние между соседними узлами сетки (вычисляет шаг), веса, размерность задачи и объем вычислений (число узлов).

\subsection*{CountMultiIntegralTrapezMethodStl()}
Реализует основной алгоритм вычислений многомерных интегралов с использованием многошаговой схемы (метод трапеций). Описание принципа работы алгоритма приводилось выше.

\section{Описание реализаций}

\subsection{Последовательная реализация (SEQ)}
\subsubsection*{Особенности}
\begin{itemize}
  \item Используется простая вышеописанная реализация.
\end{itemize}

\subsection{OpenMP реализация}
\subsection*{Общая структура}
Алгоритм использует распараллеливание с помощью директивы OpenMP для ускорения расчетов. Основные этапы:

\begin{enumerate}
    \item \textbf{Разделение работы:}
    \begin{itemize}
        \item Весь диапазон узлов сетки делится на равные блоки (chunks)
        \item Каждый поток получает свой блок узлов для обработки
        \item Используется статическое распределение: блоки назначаются потокам заранее
    \end{itemize}
    
    \item \textbf{Вычисления в потоке:}
    Каждый поток для своего блока узлов:
    \begin{itemize}
        \item Преобразует линейные индексы в многомерные координаты
        \item Вычисляет значения функции в этих точках
        \item Умножает на весовые коэффициенты
        \item Суммирует результаты в локальную переменную \texttt{local\_integral}
    \end{itemize}
    
    \item \textbf{Расчет локального объема:}
    \begin{itemize}
        \item Определяет долю обработанных точек: $\frac{\text{точек в блоке}}{\text{всего точек}}$
        \item Вычисляет локальный объем: $\text{доля} \times \prod h_k$
    \end{itemize}
    
    \item \textbf{Объединение результатов:}
    \begin{itemize}
        \item После завершения всех потоков выполняется редукция (суммирование):
        \[
        \text{integral} = \sum_{\text{потоки}} \text{local\_integral}
        \]
        \[
        \text{volume} = \sum_{\text{потоки}} \text{local\_volume}
        \]
        \item Итоговый результат: $\text{res\_} = \text{integral} \times \text{volume}$
    \end{itemize}
\end{enumerate}

\subsection*{Особенности}
\begin{itemize}
    \item Простая реализация многопоточности
    \item Автоматическая балансировка нагрузки
    \item Автоматическое управление распределением итераций
    \item Статическое распределение минимизирует накладные расходы
\end{itemize}

\subsection{MPI + OMP реализация}
\subsubsection*{Общая структура}
Реализация использует гибридный подход:
\begin{itemize}
  \item \textbf{MPI} для распределения вычислений между узлами (распараллеливание по измерениям)
  \item \textbf{OpenMP} для параллелизации внутри узла (распараллеливание по точкам сетки)
  \item Метод трапеций для численного интегрирования в $N$-мерном пространстве
\end{itemize}

\subsubsection*{Функция \texttt{RunImpl} (MPI-уровень)}
\begin{itemize}
  \item \textbf{Назначение}: Организация распределённых вычислений
  \item \textbf{Алгоритм}:
  \begin{enumerate}
    \item Корневой процесс (rank=0) определяет разбиение первой размерности
    \item Распределение параметров (пределов, шагов) через \texttt{boost::mpi::broadcast}
    \item Каждый процесс вычисляет локальный интеграл для своей подобласти
    \item Редукция результатов на корневой процесс с помощью \texttt{boost::mpi::reduce}
  \end{enumerate}
  \item \textbf{Оптимизации}: Динамическое распределение нагрузки с учётом остатка
\end{itemize}

\subsubsection*{Функция \texttt{CountMultiIntegralTrapezMethodAll} (OpenMP-уровень)}
\begin{itemize}
  \item \textbf{Назначение}: Вычисление интеграла на локальной подобласти
  \item \textbf{Алгоритм}:
  \begin{enumerate}
    \item Параллельное вычисление шагов интегрирования (\texttt{\#pragma omp parallel for})
    \item Генерация весовых коэффициентов для правила трапеций
    \item Параллельный обход $N$-мерной сетки с редукцией (\texttt{\#pragma omp parallel reduction})
    \item Вычисление итогового значения через произведение частичных результатов
  \end{enumerate}
  \item \textbf{Оптимизации}:
  \begin{itemize}
    \item Линеаризация многомерного индекса
    \item Статическое распределение нагрузки
    \item Локальность данных для потоков
  \end{itemize}
\end{itemize}

\subsubsection*{Особенности}
\begin{itemize}
  \item MPI-процессы работают на разных подобластях первой размерности
  \item Внутри каждого MPI-процесса OpenMP параллелизует вычисления по сетке
  \item Иерархия параллелизма максимизирует использование ресурсов
\end{itemize}

\subsection{TBB реализация}
\subsubsection*{Общая структура}
\begin{itemize}
  \item \textbf{Расчёт шагов интегрирования}:
  \begin{itemize}
    \item Использование \texttt{tbb::parallel\_for} для параллельного вычисления шагов $h_i$ по всем измерениям
    \item Каждый поток обрабатывает блок измерений из диапазона \texttt{blocked\_range}
  \end{itemize}
  
  \item \textbf{Генерация весовых коэффициентов}:
  \begin{itemize}
    \item Параллельное создание векторов весов для правила трапеций через \texttt{tbb::parallel\_for}
    \item Веса рассчитываются независимо для каждого измерения
  \end{itemize}
  
  \item \textbf{Вычисление интегральной суммы}:
  \begin{itemize}
    \item Применение \texttt{tbb::parallel\_reduce} для обхода всех точек $N$-мерной сетки
    \item Линеаризация многомерного индекса и расчёт вклада каждой точки
    \item Редукция частичных сумм с операцией \texttt{std::plus<>}
  \end{itemize}
  
  \item \textbf{Вычисление объёма ячейки}:
  \begin{itemize}
    \item Параллельное перемножение шагов интегрирования через \texttt{tbb::parallel\_reduce}
    \item Редукция с операцией \texttt{std::multiplies<>}
  \end{itemize}
  
  \item \textbf{Финальный результат}: $res = \text{integral} \times \text{volume}$
\end{itemize}

\subsubsection*{Особенности реализации}
\begin{itemize}
  \item Автоматическая балансировка нагрузки через work-stealing алгоритм TBB
  \item Оптимальное использование кэша благодаря блокирующим диапазонам (\texttt{blocked\_range})
  \item Минимизация накладных расходов за счёт комбинирования операций map и reduce
  \item Типобезопасные операции редукции через стандартные функторы
\end{itemize}

\subsubsection*{Преимущества подхода}
\begin{itemize}
  \item Автоматическое распределение итераций между потоками
  \item Эффективная обработка нерегулярных вычислительных нагрузок
  \item Масштабируемость на произвольное число ядер
  \item Упрощённое управление памятью по сравнению с ручной реализацией потоков
\end{itemize}

\subsection{STL threads реализация}
\subsubsection*{Общая структура}
\begin{itemize}
  \item \textbf{Подготовка данных}:
  \begin{itemize}
    \item Последовательный расчёт шагов интегрирования $h_i$ и весовых коэффициентов
    \item Вычисление общего количества точек сетки
  \end{itemize}
  
  \item \textbf{Распределение работы}:
  \begin{itemize}
    \item Статическое разбиение точек сетки на блоки (chunks)
    \item Создание пула потоков \texttt{std::thread} по числу ядер
  \end{itemize}
  
  \item \textbf{Параллельное вычисление}:
  \begin{itemize}
    \item Каждый поток обрабатывает свой диапазон точек
    \item Вычисление частичной интегральной суммы и доли объёма
    \item Линеаризация многомерных индексов
  \end{itemize}
  
  \item \textbf{Сбор результатов}:
  \begin{itemize}
    \item Синхронизация потоков через \texttt{thread.join()}
    \item Суммирование частичных интегралов и объёмов
    \item Итоговый расчёт: $res = \text{integral} \times \text{volume}$
  \end{itemize}
\end{itemize}

\subsubsection*{Особенности реализации}
\begin{itemize}
  \item Ручное управление потоками через \texttt{std::thread}
  \item Статическая балансировка нагрузки (равномерное распределение точек)
  \item Явная синхронизация с использованием \texttt{join()}
  \item Хранение результатов потоков в разделяемом векторе \texttt{thread\_results}
\end{itemize}

\subsubsection*{Преимущества}
\begin{itemize}
  \item Полный контроль над созданием и управлением потоками
  \item Минимальные зависимости (только стандартная библиотека C++)
  \item Прозрачная модель выполнения
  \item Эффективность для детерминированных нагрузок
\end{itemize}

\section{Результаты экспериментов}
\subsection*{Конфигурация системы}
\begin{itemize}
  \item CPU: Intel Core i5 13400F
  \item RAM: 32 GB DDR5 4800 Hz
  \item ОС: Windows 10 Pro
  \item Потоки: 4
  \item Процессы: 4
\end{itemize}

\subsection*{Тестовые параметры интегрирования}
\begin{itemize}
  \item \textbf{Интегрируемая функция}:
  \[
  f(x, y, z) = \sin(x) \cdot \tan(y) \cdot \ln(z)
  \]
  
  \item \textbf{Пределы интегрирования}:
  \begin{align*}
  x & \in [0.8, 1.0] \\
  y & \in [1.9, 2.0] \\
  z & \in [2.9, 3.0]
  \end{align*}
  
  \item \textbf{Числа разбиений по осям}:
  \[
  n_x = n_y = n_z = 220
  \]
  
  \item \textbf{Общее количество точек}:
  \[
  N_{\text{total}} = 221 \times 221 \times 221 = 10793861
  \]
  \item \textbf{Метод проверки}:
  \begin{itemize}
    \item Использование макроса \texttt{ASSERT\_NEAR} из тестовых фреймворков
    \item Сравнение вычисленного результата $I_{\text{calc}}$ с эталоном $I_{\text{ref}}$
    \item Проверка в пределах заданной точности $\epsilon$
  \end{itemize}
\end{itemize}

\subsection*{Производительность}
\begin{tabular}{|l|c|c|}
\hline
\textbf{Реализация} & \textbf{Pipeline (ms)} & \textbf{Task (ms)} \\
\hline
SEQ       & 2956 & 2938 \\
OMP       & 873 & 814 \\
TBB       & 917 & 849 \\
STL       & 950 & 912 \\
MPI+OMP   & 307 & \textbf{292} \\
\hline
\end{tabular}

\subsection*{Ускорение}
\[
S_{\text{OMP}} = \frac{2956}{873} \approx 3.38, \quad
S_{\text{TBB}} = \frac{2956}{917} \approx 3.22, \quad
S_{\text{STL}} = \frac{2956}{950} \approx 3.11, \quad
S_{\text{MPI+OMP}} = \frac{2956}{292} \approx 10.12
\]

\section{Выводы}
\begin{itemize}
  \item MPI+OMP показала наилучшее ускорение благодаря двухуровневой параллелизации
  \item STL Threads требует больше кода, но даёт контроль
  \item TBB и OpenMP обеспечивают простоту и масштабируемость
\end{itemize}

\section*{Список литературы}
\begin{enumerate}
  \item OpenMP Architecture Review Board. OpenMP API Specification for Parallel Programming. \url{https://www.openmp.org/}
  \item Intel. Threading Building Blocks. \url{https://www.intel.com/content/www/us/en/developer/tools/oneapi/onetbb.html}
  \item Boost C++ Libraries. Boost.MPI. \url{https://www.boost.org/doc/libs/release/doc/html/mpi.html}
  \item В.П. Гергель. «Теория и практика параллельных вычислений». М.: Интернет-Университет, 2007.
  \item Материалы лекций и лабораторных работ кафедры ВВСП, ННГУ им. Лобачевского
\end{enumerate}

\newpage
\appendix
\section*{Приложение: Исходные коды реализаций}
\subsection*{SEQ}
\begin{lstlisting}
#pragma once

#include <cstddef>
#include <functional>
#include <memory>
#include <utility>
#include <vector>

#include "core/task/include/task.hpp"

namespace poroshin_v_multi_integral_with_trapez_method_seq {

class TestTaskSequential : public ppc::core::Task {
 public:
  explicit TestTaskSequential(std::shared_ptr<ppc::core::TaskData> &task_data,
                              std::function<double(std::vector<double> &args)> func)
      : Task(task_data), dim_(task_data->inputs_count[0]), func_(std::move(func)) {}
  bool PreProcessingImpl() override;
  bool ValidationImpl() override;
  bool RunImpl() override;
  bool PostProcessingImpl() override;

 private:
  void CountMultiIntegralTrapezMethodSeq();
  std::vector<std::pair<double, double>> limits_;
  size_t dim_;
  std::function<double(std::vector<double> &args)> func_;
  std::vector<int> n_;
  double res_{};
};

}  // namespace poroshin_v_multi_integral_with_trapez_method_seq

#include "seq/poroshin_v_multi_integral_with_trapez_method/include/ops_seq.hpp"

#include <cstddef>
#include <vector>

void poroshin_v_multi_integral_with_trapez_method_seq::TestTaskSequential::CountMultiIntegralTrapezMethodSeq() {
  int dimensions = static_cast<int>(limits_.size());
  std::vector<double> h(dimensions);
  for (int i = 0; i < dimensions; ++i) {
    h[i] = (limits_[i].second - limits_[i].first) / n_[i];
  }

  double integral = 0.0;
  std::vector<double> vars(dimensions);

  std::vector<int> indices(dimensions, 0);
  int flag = 1;
  while (flag == 1) {
    for (int i = 0; i < dimensions; ++i) {
      vars[i] = limits_[i].first + indices[i] * h[i];
    }

    double weight = 1.0;
    for (int i = 0; i < dimensions; ++i) {
      weight *= (indices[i] == 0 || indices[i] == n_[i]) ? 0.5 : 1.0;
    }
    integral += func_(vars) * weight;

    int dim = 0;
    while (dim < dimensions) {
      indices[dim]++;
      if (indices[dim] <= n_[dim]) {
        break;
      }
      indices[dim] = 0;
      dim++;
      if (dim == dimensions) {
        flag = 0;
        break;
      }
    }
  }

  double volume = 1.0;
  for (int i = 0; i < dimensions; ++i) {
    volume *= h[i];
  }

  res_ = integral * volume;
}

bool poroshin_v_multi_integral_with_trapez_method_seq::TestTaskSequential::PreProcessingImpl() {
  n_.resize(dim_);
  limits_.resize(dim_);
  for (size_t i = 0; i < dim_; i++) {
    n_[i] = reinterpret_cast<int *>(task_data->inputs[0])[i];
    limits_[i].first = reinterpret_cast<double *>(task_data->inputs[1])[i];
    limits_[i].second = reinterpret_cast<double *>(task_data->inputs[2])[i];
  }
  res_ = 0;
  return true;
}

bool poroshin_v_multi_integral_with_trapez_method_seq::TestTaskSequential::ValidationImpl() {
  return (task_data->inputs_count[0] > 0 && task_data->outputs_count[0] == 1 && task_data->inputs_count[0] == dim_);
}

bool poroshin_v_multi_integral_with_trapez_method_seq::TestTaskSequential::RunImpl() {
  CountMultiIntegralTrapezMethodSeq();
  return true;
}

bool poroshin_v_multi_integral_with_trapez_method_seq::TestTaskSequential::PostProcessingImpl() {
  reinterpret_cast<double *>(task_data->outputs[0])[0] = res_;
  return true;
}
\end{lstlisting}

\subsection*{OpenMP}
\begin{lstlisting}
#pragma once

#include <cstddef>
#include <functional>
#include <memory>
#include <utility>
#include <vector>

#include "core/task/include/task.hpp"

namespace poroshin_v_multi_integral_with_trapez_method_omp {

class TestTaskOpenMP : public ppc::core::Task {
 public:
  explicit TestTaskOpenMP(std::shared_ptr<ppc::core::TaskData> &task_data,
                          std::function<double(std::vector<double> &args)> func)
      : Task(task_data), dim_(task_data->inputs_count[0]), func_(std::move(func)) {}
  bool PreProcessingImpl() override;
  bool ValidationImpl() override;
  bool RunImpl() override;
  bool PostProcessingImpl() override;

 private:
  void CountMultiIntegralTrapezMethodOmp();
  std::vector<std::pair<double, double>> limits_;
  size_t dim_;
  std::function<double(std::vector<double> &args)> func_;
  std::vector<int> n_;
  double res_{};
};

}  // namespace poroshin_v_multi_integral_with_trapez_method_omp

#include "omp/poroshin_v_multi_integral_with_trapez_method/include/ops_omp.hpp"

#include <omp.h>

#include <cmath>
#include <cstddef>
#include <functional>
#include <vector>

void poroshin_v_multi_integral_with_trapez_method_omp::TestTaskOpenMP::CountMultiIntegralTrapezMethodOmp() {
  const int dimensions = static_cast<int>(limits_.size());
  std::vector<double> h(dimensions);

#pragma omp parallel for schedule(static)
  for (int i = 0; i < dimensions; ++i) {
    h[i] = (limits_[i].second - limits_[i].first) / n_[i];
  }

  std::vector<std::vector<double>> weights(dimensions);
#pragma omp parallel for schedule(static)
  for (int i = 0; i < dimensions; ++i) {
    weights[i].resize(n_[i] + 1);
    for (int j = 0; j <= n_[i]; ++j) {
      weights[i][j] = (j == 0 || j == n_[i]) ? 0.5 : 1.0;
    }
  }

  double integral = 0.0;

#pragma omp parallel reduction(+ : integral)
{
    std::vector<double> vars(dimensions);
    std::vector<int> indices(dimensions, 0);

    int total_points = 1;
    for (int n : n_) {
      total_points *= (n + 1);
    }

#pragma omp for schedule(static)
    for (int linear_idx = 0; linear_idx < total_points; ++linear_idx) {
      int idx = linear_idx;
      for (int dim = dimensions - 1; dim >= 0; --dim) {
        indices[dim] = idx % (n_[dim] + 1);
        idx /= (n_[dim] + 1);
      }

      double weight = 1.0;
      for (int dim = 0; dim < dimensions; ++dim) {
        vars[dim] = limits_[dim].first + indices[dim] * h[dim];
        weight *= weights[dim][indices[dim]];
      }

      integral += func_(vars) * weight;
    }
  }

  double volume = 1.0;
#pragma omp parallel for reduction(* : volume)
  for (int i = 0; i < dimensions; ++i) {
    volume *= h[i];
  }

  res_ = integral * volume;
}

bool poroshin_v_multi_integral_with_trapez_method_omp::TestTaskOpenMP::PreProcessingImpl() {
  n_.resize(dim_);
  limits_.resize(dim_);
  for (size_t i = 0; i < dim_; i++) {
    n_[i] = reinterpret_cast<int *>(task_data->inputs[0])[i];
    limits_[i].first = reinterpret_cast<double *>(task_data->inputs[1])[i];
    limits_[i].second = reinterpret_cast<double *>(task_data->inputs[2])[i];
  }
  res_ = 0;
  return true;
}

bool poroshin_v_multi_integral_with_trapez_method_omp::TestTaskOpenMP::ValidationImpl() {
  return (task_data->inputs_count[0] > 0 && task_data->outputs_count[0] == 1 && task_data->inputs_count[0] == dim_);
}

bool poroshin_v_multi_integral_with_trapez_method_omp::TestTaskOpenMP::RunImpl() {
  CountMultiIntegralTrapezMethodOmp();
  return true;
}

bool poroshin_v_multi_integral_with_trapez_method_omp::TestTaskOpenMP::PostProcessingImpl() {
  reinterpret_cast<double *>(task_data->outputs[0])[0] = res_;
  return true;
}
\end{lstlisting}

\subsection*{TBB}
\begin{lstlisting}
#pragma once

#include <cstddef>
#include <functional>
#include <memory>
#include <utility>
#include <vector>

#include "core/task/include/task.hpp"

namespace poroshin_v_multi_integral_with_trapez_method_tbb {

class TestTaskTBB : public ppc::core::Task {
 public:
  explicit TestTaskTBB(std::shared_ptr<ppc::core::TaskData> &task_data,
                       std::function<double(std::vector<double> &args)> func)
      : Task(task_data), dim_(task_data->inputs_count[0]), func_(std::move(func)) {}
  bool PreProcessingImpl() override;
  bool ValidationImpl() override;
  bool RunImpl() override;
  bool PostProcessingImpl() override;

 private:
  void CountMultiIntegralTrapezMethodTbb();
  std::vector<std::pair<double, double>> limits_;
  size_t dim_;
  std::function<double(std::vector<double> &args)> func_;
  std::vector<int> n_;
  double res_{};
};

}  // namespace poroshin_v_multi_integral_with_trapez_method_tbb

#include "tbb/poroshin_v_multi_integral_with_trapez_method/include/ops_tbb.hpp"

#include <oneapi/tbb/blocked_range.h>
#include <oneapi/tbb/parallel_for.h>
#include <oneapi/tbb/parallel_reduce.h>

#include <cmath>
#include <cstddef>
#include <functional>
#include <vector>

void poroshin_v_multi_integral_with_trapez_method_tbb::TestTaskTBB::CountMultiIntegralTrapezMethodTbb() {
  const int dimensions = static_cast<int>(limits_.size());
  std::vector<double> h(dimensions);

  tbb::parallel_for(tbb::blocked_range<int>(0, dimensions), [&](const tbb::blocked_range<int>& range) {
    for (int i = range.begin(); i != range.end(); ++i) {
      h[i] = (limits_[i].second - limits_[i].first) / n_[i];
    }
  });

  std::vector<std::vector<double>> weights(dimensions);
  tbb::parallel_for(tbb::blocked_range<int>(0, dimensions), [&](const tbb::blocked_range<int>& range) {
    for (int i = range.begin(); i != range.end(); ++i) {
      weights[i].resize(n_[i] + 1);
      for (int j = 0; j <= n_[i]; ++j) {
        weights[i][j] = (j == 0 || j == n_[i]) ? 0.5 : 1.0;
      }
    }
  });

  double integral = tbb::parallel_reduce(
      tbb::blocked_range<int>(0,
                              [&]() {
                                int total_points = 1;
                                for (int n : n_) {
                                  total_points *= (n + 1);
                                }
                                return total_points;
                              }()),
      0.0,
      [&](const tbb::blocked_range<int>& range, double local_integral) {
        std::vector<double> vars(dimensions);
        std::vector<int> indices(dimensions, 0);

        for (int linear_idx = range.begin(); linear_idx != range.end(); ++linear_idx) {
          int idx = linear_idx;
          for (int dim = dimensions - 1; dim >= 0; --dim) {
            indices[dim] = idx % (n_[dim] + 1);
            idx /= (n_[dim] + 1);
          }

          double weight = 1.0;
          for (int dim = 0; dim < dimensions; ++dim) {
            vars[dim] = limits_[dim].first + indices[dim] * h[dim];
            weight *= weights[dim][indices[dim]];
          }

          local_integral += func_(vars) * weight;
        }
        return local_integral;
      },
      std::plus<>());

  double volume = tbb::parallel_reduce(
      tbb::blocked_range<int>(0, dimensions), 1.0,
      [&](const tbb::blocked_range<int>& range, double local_volume) {
        for (int i = range.begin(); i != range.end(); ++i) {
          local_volume *= h[i];
        }
        return local_volume;
      },
      std::multiplies<>());

  res_ = integral * volume;
}

bool poroshin_v_multi_integral_with_trapez_method_tbb::TestTaskTBB::PreProcessingImpl() {
  n_.resize(dim_);
  limits_.resize(dim_);
  for (size_t i = 0; i < dim_; i++) {
    n_[i] = reinterpret_cast<int*>(task_data->inputs[0])[i];
    limits_[i].first = reinterpret_cast<double*>(task_data->inputs[1])[i];
    limits_[i].second = reinterpret_cast<double*>(task_data->inputs[2])[i];
  }
  res_ = 0;
  return true;
}

bool poroshin_v_multi_integral_with_trapez_method_tbb::TestTaskTBB::ValidationImpl() {
  return (task_data->inputs_count[0] > 0 && task_data->outputs_count[0] == 1 && task_data->inputs_count[0] == dim_);
}

bool poroshin_v_multi_integral_with_trapez_method_tbb::TestTaskTBB::RunImpl() {
  CountMultiIntegralTrapezMethodTbb();
  return true;
}

bool poroshin_v_multi_integral_with_trapez_method_tbb::TestTaskTBB::PostProcessingImpl() {
  reinterpret_cast<double*>(task_data->outputs[0])[0] = res_;
  return true;
}
\end{lstlisting}

\subsection*{STL}
\begin{lstlisting}
#pragma once

#include <cstddef>
#include <functional>
#include <memory>
#include <utility>
#include <vector>

#include "core/task/include/task.hpp"

namespace poroshin_v_multi_integral_with_trapez_method_stl {

class TestTaskSTL : public ppc::core::Task {
 public:
  explicit TestTaskSTL(std::shared_ptr<ppc::core::TaskData>& task_data,
                       std::function<double(std::vector<double>& args)> func)
      : Task(task_data), dim_(task_data->inputs_count[0]), func_(std::move(func)) {}
  bool PreProcessingImpl() override;
  bool ValidationImpl() override;
  bool RunImpl() override;
  bool PostProcessingImpl() override;

 private:
  void CountMultiIntegralTrapezMethodStl();
  void CalculateData(std::vector<double>& h, std::vector<std::vector<double>>& weights, int& total_points,
                     const int& dimensions);
  std::vector<std::pair<double, double>> limits_;
  size_t dim_;
  std::function<double(std::vector<double>& args)> func_;
  std::vector<int> n_;
  double res_{};
};

}  // namespace poroshin_v_multi_integral_with_trapez_method_stl

#include "stl/poroshin_v_multi_integral_with_trapez_method/include/ops_stl.hpp"

#include <algorithm>
#include <cmath>
#include <cstddef>
#include <functional>
#include <thread>
#include <utility>
#include <vector>

#include "core/util/include/util.hpp"

void poroshin_v_multi_integral_with_trapez_method_stl::TestTaskSTL::CalculateData(
    std::vector<double>& h, std::vector<std::vector<double>>& weights, int& total_points, const int& dimensions) {
  for (int i = 0; i < dimensions; ++i) {
    h[i] = (limits_[i].second - limits_[i].first) / n_[i];

    total_points *= (n_[i] + 1);

    weights[i].resize(n_[i] + 1);
    for (int j = 0; j <= n_[i]; ++j) {
      weights[i][j] = (j == 0 || j == n_[i]) ? 0.5 : 1.0;
    }
  }
}

void poroshin_v_multi_integral_with_trapez_method_stl::TestTaskSTL::CountMultiIntegralTrapezMethodStl() {
  const int dimensions = static_cast<int>(limits_.size());
  const int num_threads = ppc::util::GetPPCNumThreads();
  std::vector<double> h(dimensions);
  std::vector<std::vector<double>> weights(dimensions);
  int total_points = 1;
  CalculateData(h, weights, total_points, dimensions);
  std::vector<std::pair<double, double>> thread_results(num_threads, {0.0, 0.0});
  std::vector<std::thread> threads;
  threads.reserve(num_threads);

  const int chunk_size = (total_points + num_threads - 1) / num_threads;

  for (int i = 0; i < num_threads; ++i) {
    int start = i * chunk_size;
    int end = std::min(start + chunk_size, total_points);

    if (start < end) {
      threads.emplace_back([this, dimensions, start, end, &h, &weights, total_points, &thread_results, i]() {
        double local_integral = 0.0;
        std::vector<double> vars(dimensions);
        std::vector<int> indices(dimensions, 0);

        for (int linear_idx = start; linear_idx < end; ++linear_idx) {
          int idx = linear_idx;
          for (int dim = dimensions - 1; dim >= 0; --dim) {
            indices[dim] = idx % (n_[dim] + 1);
            idx /= (n_[dim] + 1);
          }

          double weight = 1.0;
          for (int dim = 0; dim < dimensions; ++dim) {
            vars[dim] = limits_[dim].first + indices[dim] * h[dim];
            weight *= weights[dim][indices[dim]];
          }

          local_integral += func_(vars) * weight;
        }

        double chunk_fraction = static_cast<double>(end - start) / total_points;
        double local_volume = chunk_fraction;
        for (double hi : h) {
          local_volume *= hi;
        }

        thread_results[i] = {local_integral, local_volume};
      });
    }
  }

  for (auto& thread : threads) {
    thread.join();
  }

  double integral = 0.0;
  double volume = 0.0;
  for (const auto& result : thread_results) {
    integral += result.first;
    volume += result.second;
  }

  res_ = integral * volume;
}

bool poroshin_v_multi_integral_with_trapez_method_stl::TestTaskSTL::PreProcessingImpl() {
  n_.resize(dim_);
  limits_.resize(dim_);
  for (size_t i = 0; i < dim_; i++) {
    n_[i] = reinterpret_cast<int*>(task_data->inputs[0])[i];
    limits_[i].first = reinterpret_cast<double*>(task_data->inputs[1])[i];
    limits_[i].second = reinterpret_cast<double*>(task_data->inputs[2])[i];
  }
  res_ = 0;
  return true;
}

bool poroshin_v_multi_integral_with_trapez_method_stl::TestTaskSTL::ValidationImpl() {
  return (task_data->inputs_count[0] > 0 && task_data->outputs_count[0] == 1 && task_data->inputs_count[0] == dim_);
}

bool poroshin_v_multi_integral_with_trapez_method_stl::TestTaskSTL::RunImpl() {
  CountMultiIntegralTrapezMethodStl();
  return true;
}

bool poroshin_v_multi_integral_with_trapez_method_stl::TestTaskSTL::PostProcessingImpl() {
  reinterpret_cast<double*>(task_data->outputs[0])[0] = res_;
  return true;
}
\end{lstlisting}

\subsection*{MPI + OMP}
\begin{lstlisting}
#pragma once

#include <boost/mpi/collectives.hpp>
#include <boost/mpi/communicator.hpp>
#include <cstddef>
#include <functional>
#include <memory>
#include <utility>
#include <vector>

#include "core/task/include/task.hpp"

namespace poroshin_v_multi_integral_with_trapez_method_all {

class TestTaskALL : public ppc::core::Task {
 public:
  explicit TestTaskALL(std::shared_ptr<ppc::core::TaskData> &task_data,
                       std::function<double(std::vector<double> &args)> func)
      : Task(std::move(task_data)), func_(std::move(func)) {}
  bool PreProcessingImpl() override;
  bool ValidationImpl() override;
  bool RunImpl() override;
  bool PostProcessingImpl() override;

 private:
  void CountMultiIntegralTrapezMethodAll(double &res);
  std::vector<std::pair<double, double>> limits_;
  size_t dim_{};
  std::function<double(std::vector<double> &args)> func_;
  std::vector<int> n_;
  double res_{};
  boost::mpi::communicator world_;
};

}  // namespace poroshin_v_multi_integral_with_trapez_method_all

#include "all/poroshin_v_multi_integral_with_trapez_method/include/ops_all.hpp"

#include <omp.h>

#include <boost/serialization/utility.hpp>  // NOLINT(misc-include-cleaner)
#include <cmath>
#include <cstddef>
#include <functional>
#include <vector>

#include "boost/mpi/collectives/broadcast.hpp"
#include "boost/mpi/collectives/reduce.hpp"

void poroshin_v_multi_integral_with_trapez_method_all::TestTaskALL::CountMultiIntegralTrapezMethodAll(double &res) {
  const int dimensions = static_cast<int>(limits_.size());
  std::vector<double> h(dimensions);

#pragma omp parallel for schedule(static)
  for (int i = 0; i < dimensions; ++i) {
    h[i] = (limits_[i].second - limits_[i].first) / n_[i];
  }

  std::vector<std::vector<double>> weights(dimensions);
#pragma omp parallel for schedule(static)
  for (int i = 0; i < dimensions; ++i) {
    weights[i].resize(n_[i] + 1);
    for (int j = 0; j <= n_[i]; ++j) {
      weights[i][j] = (j == 0 || j == n_[i]) ? 0.5 : 1.0;
    }
  }

  double integral = 0.0;

#pragma omp parallel reduction(+ : integral)
  {
    std::vector<double> vars(dimensions);
    std::vector<int> indices(dimensions, 0);

    int total_points = 1;
    for (int n : n_) {
      total_points *= (n + 1);
    }

#pragma omp for schedule(static)
    for (int linear_idx = 0; linear_idx < total_points; ++linear_idx) {
      int idx = linear_idx;
      for (int dim = dimensions - 1; dim >= 0; --dim) {
        indices[dim] = idx % (n_[dim] + 1);
        idx /= (n_[dim] + 1);
      }

      double weight = 1.0;
      for (int dim = 0; dim < dimensions; ++dim) {
        vars[dim] = limits_[dim].first + indices[dim] * h[dim];
        weight *= weights[dim][indices[dim]];
      }

      integral += func_(vars) * weight;
    }
  }

  double volume = 1.0;
#pragma omp parallel for reduction(* : volume)
  for (int i = 0; i < dimensions; ++i) {
    volume *= h[i];
  }

  res = integral * volume;
}

bool poroshin_v_multi_integral_with_trapez_method_all::TestTaskALL::PreProcessingImpl() {
  if (world_.rank() == 0) {
    dim_ = task_data->inputs_count[0];
    n_.resize(dim_);
    limits_.resize(dim_);
    for (size_t i = 0; i < dim_; i++) {
      n_[i] = reinterpret_cast<int *>(task_data->inputs[0])[i];
      limits_[i].first = reinterpret_cast<double *>(task_data->inputs[1])[i];
      limits_[i].second = reinterpret_cast<double *>(task_data->inputs[2])[i];
    }
    res_ = 0;
  }
  return true;
}

bool poroshin_v_multi_integral_with_trapez_method_all::TestTaskALL::ValidationImpl() {
  if (world_.rank() == 0) {
    return (task_data->inputs_count[0] > 0 && task_data->outputs_count[0] == 1);
  }
  return true;
}

bool poroshin_v_multi_integral_with_trapez_method_all::TestTaskALL::RunImpl() {
  size_t len = 0;
  double step = 0;
  int delta = 0;
  int last = 0;
  double res = 0;
  if (world_.rank() == 0) {
    if ((n_[0] % world_.size()) == 0) {
      delta = n_[0] / world_.size();
      last = delta;
    } else {
      delta = n_[0] / world_.size();
      last = delta + n_[0] % world_.size();
    }
    step = (limits_[0].second - limits_[0].first) / static_cast<double>(n_[0]);
    len = dim_;
  }
  boost::mpi::broadcast(world_, step, 0);
  boost::mpi::broadcast(world_, delta, 0);
  boost::mpi::broadcast(world_, last, 0);
  boost::mpi::broadcast(world_, len, 0);
  if (world_.rank() != 0) {
    limits_.resize(len);
    n_.resize(len);
  }
  boost::mpi::broadcast(world_, limits_.data(), static_cast<int>(len), 0);
  boost::mpi::broadcast(world_, n_.data(), static_cast<int>(len), 0);
  if (world_.rank() == world_.size() - 1) {
    n_[0] = last;
    limits_[0].first += world_.rank() * step * delta;
  } else {
    n_[0] = delta;
    limits_[0].first += world_.rank() * step * delta;
    limits_[0].second = limits_[0].first + step * delta;
  }
  CountMultiIntegralTrapezMethodAll(res);
  boost::mpi::reduce(world_, res, res_, std::plus<>(), 0);
  return true;
}

bool poroshin_v_multi_integral_with_trapez_method_all::TestTaskALL::PostProcessingImpl() {
  if (world_.rank() == 0) {
    reinterpret_cast<double *>(task_data->outputs[0])[0] = res_;
  }
  return true;
}
\end{lstlisting}

\end{document}