\documentclass[12pt]{article}
\usepackage[russian]{babel}
\usepackage{graphicx} 
\usepackage{times}
\usepackage{amsmath}
\usepackage{listings}
\usepackage{xcolor}
\usepackage{tabularx}
\usepackage[T2A]{fontenc}
\usepackage[utf8]{inputenc}

\lstdefinestyle{mystyle}{
    keywordstyle=\color{purple},   
    numberstyle=\tiny\color{gray},   
    stringstyle=\color{green},    
    basicstyle=\ttfamily\footnotesize, 
    breakatwhitespace=false, 
    breaklines=true,  
    captionpos=b,   
    keepspaces=true,  
    numbers=left,      
    showspaces=false,  
    showstringspaces=false,   
    tabsize=2          
}

\title{YasakovaSparseMatrixReport}
\author{Ясакова Татьяна}
\date{Май 2025}

\begin{document}
\begin{titlepage}
\begin{center}
\textbf{МИНИСТЕРСТВО НАУКИ И ВЫСШЕГО ОБРАЗОВАНИЯ РОССИЙСКОЙ ФЕДЕРАЦИИ} \\
\textbf{Федеральное государственное автономное образовательное учреждение высшего образования \\ «Национальный исследовательский Нижегородский государственный университет им. Н.И. Лобачевского»} \\
Институт информационных технологий, математики и механики \\
\vspace{3cm}
{\Large
\textbf{Отчёт по лабораторной работе:} \\[0.5cm]
\textbf{"Умножение разреженных матриц. Элементы комплексного типа. Формат хранения матрицы – строковый (CRS)"} \\
}
\vspace{2cm}
\begin{flushright}
\textbf{Выполнила:} студентка группы 3822Б1ФИ2\\
Ясакова Татьяна \\
\vspace{0.5cm}
\textbf{Преподаватель:} \\
доцент кафедры ВВСП, к.т.н \\ Сысоев А.В.
\end{flushright}
\vspace{2.5cm}
Нижний Новгород \\
2025
\end{center}
\end{titlepage}

\begin{center}
    \section*{Введение}
\end{center}
\textbf{Разреженная матрица} — матрица, в которой большинство элементов равны нулю. Такие матрицы широко используются в научных вычислениях, машинном обучении, обработке графиков и других областях. Эффективная работа с разреженными матрицами требует специальных форматов хранения и алгоритмов обработки.

В данной работе рассматривается умножение разреженных матриц с элементами типа \texttt{complex<double>} в формате CRS (Compressed Row Storage). Этот формат эффективен для хранения и обработки матриц с малым количеством ненулевых элементов, так как позволяет значительно сократить объем используемой памяти и ускорить вычисления.

\newpage
\begin{center}
    \section*{Постановка задачи}
\end{center}
Необходимо реализовать алгоритм умножения двух разреженных матриц в формате CRS с поддержкой комплексных чисел. Основные этапы решения задачи:
\begin{enumerate}
    \item Реализовать структуру данных для хранения разреженной матрицы в формате CRS.
    \item Разработать алгоритм умножения разреженных матриц.
    \item Реализовать последовательную и параллельные версии алгоритма с использованием:
    \begin{itemize}
        \item OpenMP
        \item TBB (Threading Building Blocks)
        \item STL (std::thread)
        \item MPI (Message Passing Interface)
    \end{itemize}
    \item Провести тестирование и сравнение производительности реализаций.
\end{enumerate}

\newpage
\begin{center}
    \section*{Описание алгоритма}
\end{center}

Формат CRS (Compressed Row Storage) представляет разреженную матрицу тремя массивами:
\begin{itemize}
    \item \texttt{data} — ненулевые элементы матрицы.
    \item \texttt{columnIndices} — индексы столбцов для каждого элемента.
    \item \texttt{rowPointers} — указатели на начало каждой строки в массивах \texttt{data} и \texttt{columnIndices}.
\end{itemize}

Пример матрицы:
\[
A = \begin{pmatrix}
1 & 0 & 0 \\
0 & 2 & 3 \\
0 & 0 & 4
\end{pmatrix}
\]
В формате CRS:
\begin{itemize}
    \item \texttt{data} = [1, 2, 3, 4]
    \item \texttt{columnIndices} = [0, 1, 2, 2]
    \item \texttt{rowPointers} = [0, 1, 3, 4]
\end{itemize}

Алгоритм умножения матриц в формате CRS включает следующие шаги:
\begin{enumerate}
    \item Проверка совместимости размеров матриц.
    \item Итерация по строкам первой матрицы и столбцам второй матрицы.
    \item Поиск совпадающих элементов для вычисления скалярного произведения строки и столбца.
    \item Запись результата в результирующую матрицу.
\end{enumerate}

Сложность алгоритма зависит от количества ненулевых элементов и структуры матриц.

\newpage
\begin{center}
    \section*{Описание схемы параллельного алгоритма}
\end{center}

Параллельные версии алгоритма реализованы с использованием различных технологий:

\subsection*{OpenMP}
Используются директивы \texttt{\#pragma omp parallel} и \texttt{\#pragma omp for} для распределения итераций цикла умножения матриц между потоками. Каждый поток сохраняет результаты в локальный буфер, который затем объединяется в общий результат.

\subsection*{TBB}
Применяется \texttt{tbb::parallel\_for} для параллельного выполнения цикла. Используется \texttt{tbb::blocked\_range} для разделения работы между потоками. Результаты также собираются в общую структуру.

\subsection*{STL}
Реализация использует \texttt{std::thread} для создания потоков. Работа разделяется между потоками вручную, каждый поток обрабатывает свой диапазон строк матрицы.

\subsection*{MPI}
Используется \texttt{boost::mpi} для распределения данных между процессами. Каждый процесс вычисляет свою часть результата, который затем собирается на главном процессе с помощью \texttt{boost::mpi::gatherv}.

\newpage
\begin{center}
    \section*{Результаты экспериментов}
\end{center}
Тестирование проводилось на матрицах размерности 700x700 с плотностью заполнения 1/6. Результаты представлены в таблице:

\begin{tabular}{|c|c|c|c|c|}
    \hline
    Технология & 1 поток & 2 потока & 4 потока & 8 потоков \\ \hline
    Последовательная & 0.57 & - & - & - \\ \hline
    OpenMP & 1.02 & 0.57 & 0.37 & 0.29 \\ \hline
    TBB & 1.03 & 0.60 & 0.41 & 0.29 \\ \hline
    STL & 1.03 & 0.59 & 0.40 & 0.30 \\ \hline
    MPI + STL & 1.18 & 0.74 & 0.48 & 0.39 \\ \hline
\end{tabular}

\vspace{0.5cm}
\textbf{Выводы:}
\begin{itemize}
    \item Параллельные реализации показывают значительное ускорение по сравнению с последовательной версией.
    \item Наилучшие результаты достигаются при использовании 4-8 потоков.
    \item OpenMP демонстрирует наиболее стабильную производительность.
\end{itemize}

\newpage
\begin{center}
    \section*{Заключение}
\end{center}
В ходе работы были реализованы последовательная и параллельные версии алгоритма умножения разреженных матриц в формате CRS. Параллельные реализации показали значительное ускорение, особенно при использовании OpenMP и TBB. Результаты подтверждают эффективность применения параллельных вычислений для обработки разреженных матриц.

\newpage
\begin{center}
    \section*{Приложение}
\end{center}

\subsection*{Пример кода: структура CRS}
\begin{lstlisting}[language=C++]
struct SparseMatrixCRS {
  std::vector<Complex> data;
  std::vector<int> columnIndices;
  std::vector<int> rowPointers;
  int rowCount;
  int columnCount;

  void InsertElement(int row, Complex value, int col) {
    for (int j = rowPointers[row]; j < rowPointers[row + 1]; ++j) {
      if (columnIndices[j] == col) {
        data[j] += value;
        return;
      }
    }
    columnIndices.emplace_back(col);
    data.emplace_back(value);
    for (int i = row + 1; i <= rowCount; ++i) {
      rowPointers[i]++;
    }
  }
};
\end{lstlisting}

\end{document}