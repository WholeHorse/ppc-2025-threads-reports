\documentclass[a4paper,12pt]{article}
\usepackage[utf8]{inputenc}
\usepackage[T2A]{fontenc}
\usepackage[russian]{babel}
\usepackage{titlesec}
\titleformat{\section}{\normalfont\Large\bfseries}{\thesection.}{1em}{}
\titleformat{\subsection}{\normalfont\large\bfseries}{\thesubsection.}{1em}{}
\usepackage{tocloft}
\renewcommand{\cftsecleader}{\cftdotfill{\cftdotsep}}
\usepackage{hyperref}
\usepackage{geometry}
\geometry{top=2cm, bottom=2cm, left=2cm, right=2cm}
\setlength{\parindent}{1.25cm}
\setlength{\parskip}{0pt}
\usepackage{amsmath,amssymb}
\usepackage{listings}
\hypersetup{hidelinks}

\begin{document}
\begin{titlepage}
\begin{center}
\textbf{МИНИСТЕРСТВО НАУКИ И ВЫСШЕГО ОБРАЗОВАНИЯ РОССИЙСКОЙ ФЕДЕРАЦИИ} \\
\textbf{Федеральное государственное автономное образовательное учреждение высшего образования} \\
\textbf{«Национальный исследовательский Нижегородский государственный университет им. Н.И. Лобачевского»} \\[1cm]
\textbf{Институт информационных технологий, математики и механики }\\[0.5cm]
\textbf{Кафедра: Математического обеспечения и суперкомпьютерных технологий}\\[0.5cm]
Направление подготовки: «Программная инженерия»\\
Профиль подготовки: «Общий»\\

\vfill

Отчёт по лабораторной работе на тему:\\
{\Large
\textbf{«Повышение контраста полутонового изображения посредством линейной растяжки гистограммы»} \\
}
\vfill
\begin{flushright}
\textbf{Выполнил}:\\
студент группы 3822Б1ПР2 \\
Варфоломеев Григорий Викторович\\
\vspace{1cm}
\textbf{Преподаватель}: \\
доцент, кандидат технических наук\\
Сысоев Александр Владимирович \\
\textbf{Руководитель}: \\
Нестеров Александр Юрьевич\\
Оболенский Арсений Андреевич\\
\end{flushright}
\vfill
Нижний Новгород \\
2025
\end{center}
\end{titlepage}



\tableofcontents
\newpage


\section{Введение}
В данном отчёте рассматривается задача повышения контраста полутонового изображения путём линейной растяжки гистограммы, пиксели которого имеют тип int и находятся в пределах от 0 до 255. Основная цель работы - реализовать эффективные схемы распараллеливания данных и исследовать производительность этих способов по сравнению с последовательным вариантом.
\newpage

\section{Постановка задачи}
\hspace*{1.25em}Необходимо разработать пять версий алгоритма повышения контраста полутонового изображения путём линейной растяжки гистограммы для массивов целых чисел:
\begin{enumerate}
    \item Последовательная реализация для базовой проверки корректности и замера опорного времени.
    \item OpenMP-реализация, использующая директивы \texttt{\#pragma}.
    \item Intel TBB-реализация, использующая \texttt{tbb::parallel\_invoke} и \texttt{tbb::parallel\_for}.
    \item STL-реализация через \texttt{std::thread::hardware\_concurrency()}.
    \item Комбинированная MPI+OpenMP-версия, в которой Boost.MPI распределяет исходный массив по процессам, а внутри каждого процесса используется OpenMP.
\end{enumerate}

\hspace*{1.25em}Каждую версию необходимо протестировать на одном и том же наборе данных, замерить время выполнения двух тестов:
\begin{itemize}
    \item \texttt{test\_pipeline\_run}
    \item \texttt{test\_task\_run}
\end{itemize}
\newpage


\section{Описание алгоритма}
Мы получаем матрицу в виде одномерного \textit{std::vector<int>} и каждый его элемент преобразуется в новый по формуле:
\begin{center} 
{\Large 
$
\text{\normalsize newPixel} = \frac{(\text{oldPixel} - \text{min\_value}) \cdot 
(\text{max\_range} - \text{min\_range})}{\text{max\_value} - \text{min\_value}} + \text{\normalsize min\_range}
$, 
}
\end{center}
где \begin{itemize}
  \item newPixel - новое значение пикселя,
  \item oldPixel - исходное значение пикселя,
  \item max\_value - пиксель с самой высоким значением интенсивности в изображении
  \item min\_value - пиксель с самой низким значением интенсивности в изображении
  \item max\_range - верхняя граница интенсивности, в данном случае 255,
  \item min\_range - нижняя граница интенсивности, в данном случае 0.
\end{itemize}
\newpage


\section{Схемы параллельного алгоритма}
\subsection{Seq-версия}
\hspace*{1.25em}В последовательной реализации отсутствует распараллеливание. Алгоритм выполняет:
\begin{enumerate}
    \item Чтение входных данных в вектор \texttt{input\_}.
    \item Растяжение пикселей по гистограмме при $n>1$.
    \item Копирование отсортированного результата в выходной буфер.
\end{enumerate}
\noindent\textbf{Ключевые детали реализации:}

     \begin{lstlisting}[language=C++,
    breaklines=true,       % Автоматический перенос строк
    columns=fullflexible ]
uint8_t min = *std::ranges::min_element(img_);
  uint8_t max = *std::ranges::max_element(img_);

  if (max != min) {
    for (size_t i = 0; i < img_.size(); i++) {
      res_[i] = static_cast<uint8_t>(((img_[i] - min) * 255 + (max - min) / 2) / (max - min));
    }
  } else {
    std::ranges::fill(res_.begin(), res_.end(), min);
  }
    \end{lstlisting}


\subsection{OpenMP-версия}
\hspace*{1.25em}В OMP реализации распараллеливание заключается в добавлении парадигм \texttt{\#pragma omp} в доступные для распараллеливания участки кода. Алгоритм так же выполняет:
\begin{enumerate}
    \item Чтение входных данных в вектор \texttt{input\_}.
      \item Растяжение пикселей по гистограмме при $n>1$.
    \item Копирование отсортированного результата в выходной буфер.
\end{enumerate}


\subsection{TBB-версия}
\hspace*{1.25em}В TBB реализации распараллеливание заключается в добавлении \texttt{tbb::parallel\_invoke} и \texttt{tbb::parallel\_for.} в доступные для распараллеливания участки кода. Алгоритм так же как и Seq-версия выполняет:
\begin{enumerate}
    \item Чтение входных данных в вектор \texttt{input\_}.
      \item Растяжение пикселей по гистограмме при $n>1$.
    \item Копирование отсортированного результата в выходной буфер.
\end{enumerate}
\subsection{STL-версия}
\hspace*{1.25em}В STL реализации распараллеливание достигается с помощью механизма потоков стандартной библиотеки C++. Алгоритм выполняет:
\begin{enumerate}
\item Чтение входных данных в вектор \texttt{img\_}
\item Определение минимального и максимального значений пикселей
\item Распараллеливание обработки изображения с использованием:
\begin{itemize}
\item \texttt{std::thread} для создания рабочих потоков
\item Разделение данных на блоки по числу доступных потоков
\item Параллельное применение формулы растяжения гистограммы к каждому блоку
\end{itemize}
\item Синхронизация потоков с помощью \texttt{thread.join()}
\item Копирование результата в выходной буфер
\end{enumerate}
\newpage


\section{Результаты экспериментов}
Для оценки производительности последовательной и параллельной версий алгоритма использовалась следующая тестовая система:
\begin{itemize}
    \item Операционная система: Windows 10
    \item Процессор: 13th Gen Intel(R) Core(TM) i7-13700K, 3400 МГц, ядер: 16, логических процессоров: 24
    \item Оперативная память: 32 ГБ.
\end{itemize}
На основе выполнения тестов для последовательной и параллельной (на 4 потоках для OMP,TBB,STL и на 2,3,4 процессах для MPI+OMP) версиях была создана следующая таблица:\\[0,5cm]
\begin{tabular}{|c|c|c|c|}
    \hline
    Число процессов & Вид теста      & Время выполнения (с) & Ускорение \\ \hline
    1               & pipeline\_run\_seq & 0.731                & 1.00      \\ \hline
    1               & task\_run\_seq     & 0.493                & 1.00      \\ \hline
    1               & pipeline\_run\_omp & 0.048                & 15.23      \\ \hline
    1               & task\_run\_omp     & 0.018                & 27.39      \\ \hline
    1               & pipeline\_run\_tbb & 0.259                & 2.822      \\ \hline
    1               & task\_run\_tbb     & 0.237                & 2.080      \\ \hline
    1               & pipeline\_run\_stl & 0.133                & 5.50      \\ \hline
    1               & task\_run\_stl     & 0.100                & 4.93      \\ \hline
    2               & pipeline\_run & 0.168                & 4.35      \\ \hline
    2               & task\_run     & 0.120                & 4.11      \\ \hline
    3               & pipeline\_run & 0.185                & 3.95      \\ \hline
    3               & task\_run     & 0.164                & 3.00      \\ \hline
    4               & pipeline\_run & 0.164                & 4,457      \\ \hline
    4               & task\_run     & 0.125                & 3.944      \\ \hline
\end{tabular}
\newpage

\section{Выводы из результатов}
На основе результатов экспериментов можно сделать вывод, что данная задача подходит для распараллеливания. В разы выгоднее эту задачу решать через OMP-версию программы.
\newpage
\section{Заключение}
В ходе выполнения лабораторной работы были успешно реализованы и исследованы пять версий алгоритма повышения контраста полутонового изображения методом линейной растяжки гистограммы:

\begin{itemize}
    \item Последовательная реализация (Seq)
    \item Параллельная реализация с использованием OpenMP
    \item Параллельная реализация с использованием Intel TBB
    \item Параллельная реализация с использованием STL (std::thread)
    \item Гибридная реализация с использованием MPI + OpenMP
\end{itemize}

\subsection{Основные результаты}
Экспериментальные исследования показали значительное ускорение параллельных версий по сравнению с последовательной реализацией:

\begin{itemize}
    \item OpenMP-версия продемонстрировала максимальное ускорение - до 27.39 раз
    \item STL-версия показала ускорение до 5.5 раз
    \item TBB-версия обеспечила ускорение до 2.82 раз
    \item Гибридная MPI+OpenMP реализация показала ускорение до 4.46 раз
\end{itemize}

\subsection{Выводы}
\begin{enumerate}
    \item Наибольшую эффективность показала OpenMP-реализация благодаря низким накладным расходам на создание потоков
    \item Алгоритм хорошо поддаётся параллелизации благодаря независимости обработки отдельных пикселей
    \item Оптимальное количество потоков для данной задачи - 4-8 (по числу физических ядер процессора)
    \item Гибридный подход MPI+OpenMP показал хорошую масштабируемость для больших изображений
\end{enumerate}

\subsection{Перспективы развития}
Дальнейшие исследования могут быть направлены на:
\begin{itemize}
    \item Оптимизацию работы с памятью для уменьшения времени доступа
    \item Исследование эффективности на GPU с использованием CUDA/OpenCL
    \item Разработку адаптивного алгоритма выбора оптимального числа потоков
    \item Применение более сложных методов контрастирования гистограммы
\end{itemize}


Практическая значимость работы подтверждается возможностью применения разработанных алгоритмов в системах обработки медицинских изображений, системах видеонаблюдения и других областях компьютерного зрения.
\newpage

\section{Список литературы}
\begin{thebibliography}{9}
\bibitem{Gonzalez2018}
  R. C. Gonzalez, R. E. Woods, \emph{Digital Image Processing}, 4th Edition, Pearson, 2018.

\bibitem{Szeliski2010}
  R. Szeliski, \emph{Computer Vision: Algorithms and Applications}, Springer, 2010.

\bibitem{OpenMP}
  OpenMP Architecture Review Board, \emph{OpenMP Application Programming Interface Specification}, Version 5.2, 2021.

\bibitem{TBBGuide}
  Intel Corporation, \emph{Intel® oneAPI Threading Building Blocks Developer Guide}, 2023.

\bibitem{MPIStandard}
  MPI Forum, \emph{MPI: A Message-Passing Interface Standard}, Version 4.0, 2021.

\bibitem{CppConcurrency}
  A. Williams, \emph{C++ Concurrency in Action}, 2nd Edition, Manning, 2019.

\bibitem{ParallelProgramming}
  P. Pacheco, \emph{An Introduction to Parallel Programming}, 2nd Edition, Morgan Kaufmann, 2022.

\bibitem{ImageProcessingAlgos}
  W. Burger, M. J. Burge, \emph{Digital Image Processing: An Algorithmic Introduction}, 3rd Edition, Springer, 2022.

\bibitem{Russ2016}
  J. C. Russ, \emph{The Image Processing Handbook}, 7th Edition, CRC Press, 2016.
\end{thebibliography}
\newpage
\section{Приложение}
\subsection{ops\_seq.cpp}
\begin{lstlisting}[language=C++,
    breaklines=true,       % Автоматический перенос строк
    columns=fullflexible ]
#include "seq/varfolomeev_g_histogram_linear_stretching/include/ops_seq.hpp"

#include <algorithm>
#include <cmath>
#include <cstddef>
#include <cstdint>
#include <vector>

bool varfolomeev_g_histogram_linear_stretching_seq::TestTaskSequential::PreProcessingImpl() {
  // Init value for input and output
  unsigned int input_size = task_data->inputs_count[0];
  auto *in_ptr = reinterpret_cast<uint8_t *>(task_data->inputs[0]);
  img_ = std::vector<uint8_t>(in_ptr, in_ptr + input_size);
  res_.resize(img_.size());
  return true;
}

bool varfolomeev_g_histogram_linear_stretching_seq::TestTaskSequential::ValidationImpl() {
  // Check equality of counts elements
  return task_data->inputs_count[0] != 0 && task_data->inputs_count[0] == task_data->outputs_count[0];
}

bool varfolomeev_g_histogram_linear_stretching_seq::TestTaskSequential::RunImpl() {
  uint8_t min = *std::ranges::min_element(img_);
  uint8_t max = *std::ranges::max_element(img_);

  if (max != min) {
    for (size_t i = 0; i < img_.size(); i++) {
      res_[i] = static_cast<uint8_t>(((img_[i] - min) * 255 + (max - min) / 2) / (max - min));
    }
  } else {
    std::ranges::fill(res_.begin(), res_.end(), min);
  }
  return true;
}

bool varfolomeev_g_histogram_linear_stretching_seq::TestTaskSequential::PostProcessingImpl() {
  for (size_t i = 0; i < res_.size(); i++) {
    reinterpret_cast<uint8_t *>(task_data->outputs[0])[i] = res_[i];
  }
  return true;
}
\end{lstlisting}

\subsection{ops\_omp.cpp}
\begin{lstlisting}[language=C++,
    breaklines=true,       % Автоматический перенос строк
    basicstyle=\small\ttfamily, % Уменьшенный шрифт
    columns=fullflexible ]
    #include "omp/varfolomeev_g_histogram_linear_stretching/include/ops_omp.hpp"

#include <omp.h>

#include <algorithm>
#include <cmath>
#include <cstdint>
#include <cstring>
#include <vector>

bool varfolomeev_g_histogram_linear_stretching_omp::TestTaskSequential::PreProcessingImpl() {
  // Init value for input and output
  unsigned int input_size = task_data->inputs_count[0];
  auto *in_ptr = reinterpret_cast<uint8_t *>(task_data->inputs[0]);
  img_ = std::vector<uint8_t>(in_ptr, in_ptr + input_size);
  res_.resize(img_.size());
  return true;
}

bool varfolomeev_g_histogram_linear_stretching_omp::TestTaskSequential::ValidationImpl() {
  return task_data->outputs_count[0] > 0 && task_data->inputs_count[0] > 0 &&
         task_data->inputs_count[0] == task_data->outputs_count[0];
}


bool varfolomeev_g_histogram_linear_stretching_omp::TestTaskSequential::RunImpl() {
  uint8_t min = *std::ranges::min_element(img_);
  uint8_t max = *std::ranges::max_element(img_);

  if (max != min) {
#pragma omp parallel for
    for (int i = 0; i < static_cast<int>(img_.size()); i++) {
      res_[i] = static_cast<uint8_t>(((img_[i] - min) * 255 + (max - min) / 2) / (max - min));
    }
  } else {
    std::ranges::fill(res_.begin(), res_.end(), min);
  }
  return true;
}

bool varfolomeev_g_histogram_linear_stretching_omp::TestTaskSequential::PostProcessingImpl() {
  std::memcpy(task_data->outputs[0], res_.data(), res_.size() * sizeof(uint8_t));

  return true;
}
    \end{lstlisting}
\subsection{ops\_tbb.cpp}
\begin{lstlisting}[language=C++,
    breaklines=true,       % Автоматический перенос строк
    basicstyle=\small\ttfamily, % Уменьшенный шрифт
    columns=fullflexible ]
   #include "tbb/varfolomeev_g_histogram_linear_stretching/include/ops_tbb.hpp"

#include <algorithm>
#include <cmath>
#include <cstdint>
#include <cstring>
#include <vector>

#include "oneapi/tbb/parallel_for.h"

bool varfolomeev_g_histogram_linear_stretching_tbb::TestTaskTBB ::PreProcessingImpl() {
  // Init value for input and output
  unsigned int input_size = task_data->inputs_count[0];
  auto *in_ptr = reinterpret_cast<uint8_t *>(task_data->inputs[0]);
  img_ = std::vector<uint8_t>(in_ptr, in_ptr + input_size);
  res_.resize(img_.size());
  return true;
}

bool varfolomeev_g_histogram_linear_stretching_tbb::TestTaskTBB ::ValidationImpl() {
  return task_data->outputs_count[0] > 0 && task_data->inputs_count[0] > 0 &&
         task_data->inputs_count[0] == task_data->outputs_count[0];
}

bool varfolomeev_g_histogram_linear_stretching_tbb::TestTaskTBB ::RunImpl() {
  uint8_t min_val = *std::ranges::min_element(img_);
  uint8_t max_val = *std::ranges::max_element(img_);

  if (max_val != min_val) {
    tbb::parallel_for(tbb::blocked_range<size_t>(0, img_.size()), [&](const tbb::blocked_range<size_t> &r) {
      for (size_t i = r.begin(); i != r.end(); ++i) {
        res_[i] = static_cast<uint8_t>(std::round(static_cast<double>(img_[i] - min_val) * 255 / (max_val - min_val)));
      }
    });
  } else {
    std::ranges::fill(res_.begin(), res_.end(), min_val);
  }
  return true;
}

bool varfolomeev_g_histogram_linear_stretching_tbb::TestTaskTBB ::PostProcessingImpl() {
  std::memcpy(task_data->outputs[0], res_.data(), res_.size() * sizeof(uint8_t));

  return true;
}
    \end{lstlisting}
\subsection{ops\_stl.cpp}
\begin{lstlisting}[language=C++,
    breaklines=true,       % Автоматический перенос строк
    basicstyle=\small\ttfamily, % Уменьшенный шрифт
    columns=fullflexible ]
    #include "stl/varfolomeev_g_histogram_linear_stretching/include/ops_stl.hpp"

#include <algorithm>
#include <cmath>
#include <cstdint>
#include <cstring>
#include <thread>
#include <vector>

#include "core/util/include/util.hpp"

bool varfolomeev_g_histogram_linear_stretching_stl::TestTaskSTL ::PreProcessingImpl() {
  // Init value for input and output
  unsigned int input_size = task_data->inputs_count[0];
  auto *in_ptr = reinterpret_cast<uint8_t *>(task_data->inputs[0]);
  img_ = std::vector<uint8_t>(in_ptr, in_ptr + input_size);
  res_.resize(img_.size());
  return true;
}

bool varfolomeev_g_histogram_linear_stretching_stl::TestTaskSTL ::ValidationImpl() {
  return task_data->outputs_count[0] > 0 && task_data->inputs_count[0] > 0 &&
         task_data->inputs_count[0] == task_data->outputs_count[0];
}

bool varfolomeev_g_histogram_linear_stretching_stl::TestTaskSTL ::RunImpl() {
  uint8_t min_val = *std::ranges::min_element(img_);
  uint8_t max_val = *std::ranges::max_element(img_);

  if (max_val != min_val) {
    const size_t num_threads = ppc::util::GetPPCNumThreads();
    std::vector<std::thread> threads;
    const size_t chunk_size = img_.size() / num_threads;

    for (size_t t = 0; t < num_threads; ++t) {
      const size_t start = t * chunk_size;
      const size_t end = (t == num_threads - 1) ? img_.size() : start + chunk_size;


bool varfolomeev_g_histogram_linear_stretching_omp::TestTaskSequential::RunImpl() {
  uint8_t min = *std::ranges::min_element(img_);
  uint8_t max = *std::ranges::max_element(img_);

  if (max != min) {
#pragma omp parallel for
    for (int i = 0; i < static_cast<int>(img_.size()); i++) {
      res_[i] = static_cast<uint8_t>(((img_[i] - min) * 255 + (max - min) / 2) / (max - min));
    }
  } else {
    std::ranges::fill(res_.begin(), res_.end(), min);
  }
  return true;
}

bool varfolomeev_g_histogram_linear_stretching_omp::TestTaskSequential::PostProcessingImpl() {
  std::memcpy(task_data->outputs[0], res_.data(), res_.size() * sizeof(uint8_t));

  return true;
}
    \end{lstlisting}
\subsection{ops\_all.cpp}
\begin{lstlisting}[language=C++,
    breaklines=true,       % Автоматический перенос строк
    basicstyle=\small\ttfamily, % Уменьшенный шрифт
    columns=fullflexible ]
#include "all/varfolomeev_g_histogram_linear_stretching/include/ops_all.hpp"

#include <omp.h>

#include <algorithm>
#include <boost/mpi/collectives/all_reduce.hpp>
#include <boost/mpi/collectives/gatherv.hpp>
#include <boost/mpi/operations.hpp>
#include <cmath>
#include <cstdint>
#include <vector>

namespace {
void StretchHistogram(std::vector<uint8_t>& local_data, int global_min, int global_max) {
  if (global_min != global_max) {
#pragma omp parallel for
    for (int i = 0; i < static_cast<int>(local_data.size()); ++i) {
      local_data[i] =
          static_cast<uint8_t>(std::round((local_data[i] - global_min) * 255.0 / (global_max - global_min)));
    }
  }
}
}  // namespace

void varfolomeev_g_histogram_linear_stretching_all::TestTaskALL::ScatterData(std::vector<uint8_t>& local_data) {
  if (world_.rank() == 0) {
    int total_size = static_cast<int>(input_image_.size());
    int world_size = world_.size();

    std::vector<int> counts(world_.size());
    std::vector<int> displs(world_.size());

    int base_count = total_size / world_size;
    int remainder = total_size % world_size;

    for (int i = 0; i < world_size; ++i) {
      counts[i] = base_count + (i < remainder ? 1 : 0);
      displs[i] = (i == 0) ? 0 : (displs[i - 1] + counts[i - 1]);
    }

    for (int proc = 1; proc < world_.size(); ++proc) {
      std::vector<uint8_t> proc_data(input_image_.begin() + displs[proc],
                                     input_image_.begin() + displs[proc] + counts[proc]);
      world_.send(proc, 0, proc_data);
    }
    local_data.assign(input_image_.begin(), input_image_.begin() + counts[0]);
  } else {
    world_.recv(0, 0, local_data);
  }
}

void varfolomeev_g_histogram_linear_stretching_all::TestTaskALL::GatherResults(const std::vector<uint8_t>& local_data) {
  int total_size = static_cast<int>(input_image_.size());
  int world_size = world_.size();

  std::vector<int> counts(world_size);
  std::vector<int> displs(world_size);

  int base_count = total_size / world_size;
  int remainder = total_size % world_size;

  for (int i = 0; i < world_size; ++i) {
    counts[i] = base_count + (i < remainder ? 1 : 0);
    displs[i] = (i == 0) ? 0 : (displs[i - 1] + counts[i - 1]);
  }

  if (world_.rank() == 0) {
    result_image_.resize(total_size);
  }

  boost::mpi::gatherv(world_, local_data.data(), static_cast<int>(local_data.size()), result_image_.data(), counts,
                      displs, 0);
}

void varfolomeev_g_histogram_linear_stretching_all::TestTaskALL::FindMinMax(const std::vector<uint8_t>& local_data,
                                                                            int& global_min, int& global_max) {
  int local_min = 255;
  int local_max = 0;

  if (!local_data.empty()) {
    local_min = *std::ranges::min_element(local_data);
    local_max = *std::ranges::max_element(local_data);
  }

  boost::mpi::all_reduce(world_, local_min, global_min, boost::mpi::minimum<int>());
  boost::mpi::all_reduce(world_, local_max, global_max, boost::mpi::maximum<int>());

  if (global_min == global_max) {
    global_min = 0;
    global_max = 255;
  }
}


bool varfolomeev_g_histogram_linear_stretching_all::TestTaskALL::ValidationImpl() {
  if (world_.rank() == 0) {
    return task_data->inputs_count[0] == task_data->outputs_count[0] && task_data->inputs_count[0] > 0;
  }
  return true;
}

bool varfolomeev_g_histogram_linear_stretching_all::TestTaskALL::PreProcessingImpl() {
  input_image_.clear();
  result_image_.clear();
  if (world_.rank() == 0) {
    input_image_.resize(task_data->inputs_count[0]);
    auto* input_ptr = reinterpret_cast<uint8_t*>(task_data->inputs[0]);
    std::ranges::copy(input_ptr, input_ptr + task_data->inputs_count[0], input_image_.begin());
  }

  return true;
}

bool varfolomeev_g_histogram_linear_stretching_all::TestTaskALL::RunImpl() {
  std::vector<uint8_t> local_data;
  ScatterData(local_data);

  int global_min = 0;
  int global_max = 255;
  FindMinMax(local_data, global_min, global_max);

  StretchHistogram(local_data, global_min, global_max);

  GatherResults(local_data);

  return true;
}

bool varfolomeev_g_histogram_linear_stretching_all::TestTaskALL::PostProcessingImpl() {
  if (world_.rank() == 0) {
    auto* output_ptr = reinterpret_cast<uint8_t*>(task_data->outputs[0]);
    std::ranges::copy(result_image_.begin(), result_image_.end(), output_ptr);
  }
  return true;}
\end{lstlisting}

\end{document}
